\documentclass[a4paper]{article}
\usepackage{amsmath}
\usepackage{graphicx}
\usepackage{hyperref}
\usepackage[english]{babel}
\usepackage[utf8]{inputenc}

\title{Introducere}
\author{Arvinte Rares}
\date{}

\begin{document}
\maketitle


\large
\textbf{Context}\\
\normalsize

În lume există persoane cu deficiențe de vedere acestea fiind cauzate de diferiți factori neurologici sau fiziologici. În secolul 21, odată cu dezvoltarea rapidă a internetului, a apărut nevoia de a se creea condiții necesare și pentru aceste persoane să poată folosi noile tehnologii care chiar pot să îi ajute să se integreze mai ușor în societate și de asemenea să le faciliteze viața.\\

\large
\textbf{Motivație}\\
\normalsize

Această aplicație dorește să ofere persoanelor cu probleme de vedere o șansă de a avea acces la noile tehnologii, mai exact cele din mediul online fără a avea nevoie de o tastatură specială în alfabetul braile. Utilizatorii vor putea prin intermediul utilizării vocii să aibă acces la anumite facilități create de google. Utilizatorul va putea accesa mailul, asculta muzică, va putea primi definiții pentru cuvinte din dicționar, va avea acces la google drive și multe altele. Aplicația pe care o dezvolt interoghează motorul de căutare vocal. Astfel un utilizator va transmite o comandă vocală către aplicație iar aplicația va accesa motorul de căutare google în vederea obținerii informațiilor dorite.
\end{document}
